\documentclass[11pt]{article}

    \usepackage[breakable]{tcolorbox}
    \usepackage{parskip} % Stop auto-indenting (to mimic markdown behaviour)
    

    % Basic figure setup, for now with no caption control since it's done
    % automatically by Pandoc (which extracts ![](path) syntax from Markdown).
    \usepackage{graphicx}
    % Keep aspect ratio if custom image width or height is specified
    \setkeys{Gin}{keepaspectratio}
    % Maintain compatibility with old templates. Remove in nbconvert 6.0
    \let\Oldincludegraphics\includegraphics
    % Ensure that by default, figures have no caption (until we provide a
    % proper Figure object with a Caption API and a way to capture that
    % in the conversion process - todo).
    \usepackage{caption}
    \DeclareCaptionFormat{nocaption}{}
    \captionsetup{format=nocaption,aboveskip=0pt,belowskip=0pt}

    \usepackage{float}
    \floatplacement{figure}{H} % forces figures to be placed at the correct location
    \usepackage{xcolor} % Allow colors to be defined
    \usepackage{enumerate} % Needed for markdown enumerations to work
    \usepackage{geometry} % Used to adjust the document margins
    \usepackage{amsmath} % Equations
    \usepackage{amssymb} % Equations
    \usepackage{textcomp} % defines textquotesingle
    % Hack from http://tex.stackexchange.com/a/47451/13684:
    \AtBeginDocument{%
        \def\PYZsq{\textquotesingle}% Upright quotes in Pygmentized code
    }
    \usepackage{upquote} % Upright quotes for verbatim code
    \usepackage{eurosym} % defines \euro

    \usepackage{iftex}
    \ifPDFTeX
        \usepackage[T1]{fontenc}
        \IfFileExists{alphabeta.sty}{
              \usepackage{alphabeta}
          }{
              \usepackage[mathletters]{ucs}
              \usepackage[utf8x]{inputenc}
          }
    \else
        \usepackage{fontspec}
        \usepackage{unicode-math}
    \fi

    \usepackage{fancyvrb} % verbatim replacement that allows latex
    \usepackage{grffile} % extends the file name processing of package graphics
                         % to support a larger range
    \makeatletter % fix for old versions of grffile with XeLaTeX
    \@ifpackagelater{grffile}{2019/11/01}
    {
      % Do nothing on new versions
    }
    {
      \def\Gread@@xetex#1{%
        \IfFileExists{"\Gin@base".bb}%
        {\Gread@eps{\Gin@base.bb}}%
        {\Gread@@xetex@aux#1}%
      }
    }
    \makeatother
    \usepackage[Export]{adjustbox} % Used to constrain images to a maximum size
    \adjustboxset{max size={0.9\linewidth}{0.9\paperheight}}

    % The hyperref package gives us a pdf with properly built
    % internal navigation ('pdf bookmarks' for the table of contents,
    % internal cross-reference links, web links for URLs, etc.)
    \usepackage{hyperref}
    % The default LaTeX title has an obnoxious amount of whitespace. By default,
    % titling removes some of it. It also provides customization options.
    \usepackage{titling}
    \usepackage{longtable} % longtable support required by pandoc >1.10
    \usepackage{booktabs}  % table support for pandoc > 1.12.2
    \usepackage{array}     % table support for pandoc >= 2.11.3
    \usepackage{calc}      % table minipage width calculation for pandoc >= 2.11.1
    \usepackage[inline]{enumitem} % IRkernel/repr support (it uses the enumerate* environment)
    \usepackage[normalem]{ulem} % ulem is needed to support strikethroughs (\sout)
                                % normalem makes italics be italics, not underlines
    \usepackage{soul}      % strikethrough (\st) support for pandoc >= 3.0.0
    \usepackage{mathrsfs}
    

    
    % Colors for the hyperref package
    \definecolor{urlcolor}{rgb}{0,.145,.698}
    \definecolor{linkcolor}{rgb}{.71,0.21,0.01}
    \definecolor{citecolor}{rgb}{.12,.54,.11}

    % ANSI colors
    \definecolor{ansi-black}{HTML}{3E424D}
    \definecolor{ansi-black-intense}{HTML}{282C36}
    \definecolor{ansi-red}{HTML}{E75C58}
    \definecolor{ansi-red-intense}{HTML}{B22B31}
    \definecolor{ansi-green}{HTML}{00A250}
    \definecolor{ansi-green-intense}{HTML}{007427}
    \definecolor{ansi-yellow}{HTML}{DDB62B}
    \definecolor{ansi-yellow-intense}{HTML}{B27D12}
    \definecolor{ansi-blue}{HTML}{208FFB}
    \definecolor{ansi-blue-intense}{HTML}{0065CA}
    \definecolor{ansi-magenta}{HTML}{D160C4}
    \definecolor{ansi-magenta-intense}{HTML}{A03196}
    \definecolor{ansi-cyan}{HTML}{60C6C8}
    \definecolor{ansi-cyan-intense}{HTML}{258F8F}
    \definecolor{ansi-white}{HTML}{C5C1B4}
    \definecolor{ansi-white-intense}{HTML}{A1A6B2}
    \definecolor{ansi-default-inverse-fg}{HTML}{FFFFFF}
    \definecolor{ansi-default-inverse-bg}{HTML}{000000}

    % common color for the border for error outputs.
    \definecolor{outerrorbackground}{HTML}{FFDFDF}

    % commands and environments needed by pandoc snippets
    % extracted from the output of `pandoc -s`
    \providecommand{\tightlist}{%
      \setlength{\itemsep}{0pt}\setlength{\parskip}{0pt}}
    \DefineVerbatimEnvironment{Highlighting}{Verbatim}{commandchars=\\\{\}}
    % Add ',fontsize=\small' for more characters per line
    \newenvironment{Shaded}{}{}
    \newcommand{\KeywordTok}[1]{\textcolor[rgb]{0.00,0.44,0.13}{\textbf{{#1}}}}
    \newcommand{\DataTypeTok}[1]{\textcolor[rgb]{0.56,0.13,0.00}{{#1}}}
    \newcommand{\DecValTok}[1]{\textcolor[rgb]{0.25,0.63,0.44}{{#1}}}
    \newcommand{\BaseNTok}[1]{\textcolor[rgb]{0.25,0.63,0.44}{{#1}}}
    \newcommand{\FloatTok}[1]{\textcolor[rgb]{0.25,0.63,0.44}{{#1}}}
    \newcommand{\CharTok}[1]{\textcolor[rgb]{0.25,0.44,0.63}{{#1}}}
    \newcommand{\StringTok}[1]{\textcolor[rgb]{0.25,0.44,0.63}{{#1}}}
    \newcommand{\CommentTok}[1]{\textcolor[rgb]{0.38,0.63,0.69}{\textit{{#1}}}}
    \newcommand{\OtherTok}[1]{\textcolor[rgb]{0.00,0.44,0.13}{{#1}}}
    \newcommand{\AlertTok}[1]{\textcolor[rgb]{1.00,0.00,0.00}{\textbf{{#1}}}}
    \newcommand{\FunctionTok}[1]{\textcolor[rgb]{0.02,0.16,0.49}{{#1}}}
    \newcommand{\RegionMarkerTok}[1]{{#1}}
    \newcommand{\ErrorTok}[1]{\textcolor[rgb]{1.00,0.00,0.00}{\textbf{{#1}}}}
    \newcommand{\NormalTok}[1]{{#1}}

    % Additional commands for more recent versions of Pandoc
    \newcommand{\ConstantTok}[1]{\textcolor[rgb]{0.53,0.00,0.00}{{#1}}}
    \newcommand{\SpecialCharTok}[1]{\textcolor[rgb]{0.25,0.44,0.63}{{#1}}}
    \newcommand{\VerbatimStringTok}[1]{\textcolor[rgb]{0.25,0.44,0.63}{{#1}}}
    \newcommand{\SpecialStringTok}[1]{\textcolor[rgb]{0.73,0.40,0.53}{{#1}}}
    \newcommand{\ImportTok}[1]{{#1}}
    \newcommand{\DocumentationTok}[1]{\textcolor[rgb]{0.73,0.13,0.13}{\textit{{#1}}}}
    \newcommand{\AnnotationTok}[1]{\textcolor[rgb]{0.38,0.63,0.69}{\textbf{\textit{{#1}}}}}
    \newcommand{\CommentVarTok}[1]{\textcolor[rgb]{0.38,0.63,0.69}{\textbf{\textit{{#1}}}}}
    \newcommand{\VariableTok}[1]{\textcolor[rgb]{0.10,0.09,0.49}{{#1}}}
    \newcommand{\ControlFlowTok}[1]{\textcolor[rgb]{0.00,0.44,0.13}{\textbf{{#1}}}}
    \newcommand{\OperatorTok}[1]{\textcolor[rgb]{0.40,0.40,0.40}{{#1}}}
    \newcommand{\BuiltInTok}[1]{{#1}}
    \newcommand{\ExtensionTok}[1]{{#1}}
    \newcommand{\PreprocessorTok}[1]{\textcolor[rgb]{0.74,0.48,0.00}{{#1}}}
    \newcommand{\AttributeTok}[1]{\textcolor[rgb]{0.49,0.56,0.16}{{#1}}}
    \newcommand{\InformationTok}[1]{\textcolor[rgb]{0.38,0.63,0.69}{\textbf{\textit{{#1}}}}}
    \newcommand{\WarningTok}[1]{\textcolor[rgb]{0.38,0.63,0.69}{\textbf{\textit{{#1}}}}}


    % Define a nice break command that doesn't care if a line doesn't already
    % exist.
    \def\br{\hspace*{\fill} \\* }
    % Math Jax compatibility definitions
    \def\gt{>}
    \def\lt{<}
    \let\Oldtex\TeX
    \let\Oldlatex\LaTeX
    \renewcommand{\TeX}{\textrm{\Oldtex}}
    \renewcommand{\LaTeX}{\textrm{\Oldlatex}}
    % Document parameters
    % Document title
    \title{Lesson\_00\_details\_about\_the\_exam}
    
    
    
    
    
    
    
% Pygments definitions
\makeatletter
\def\PY@reset{\let\PY@it=\relax \let\PY@bf=\relax%
    \let\PY@ul=\relax \let\PY@tc=\relax%
    \let\PY@bc=\relax \let\PY@ff=\relax}
\def\PY@tok#1{\csname PY@tok@#1\endcsname}
\def\PY@toks#1+{\ifx\relax#1\empty\else%
    \PY@tok{#1}\expandafter\PY@toks\fi}
\def\PY@do#1{\PY@bc{\PY@tc{\PY@ul{%
    \PY@it{\PY@bf{\PY@ff{#1}}}}}}}
\def\PY#1#2{\PY@reset\PY@toks#1+\relax+\PY@do{#2}}

\@namedef{PY@tok@w}{\def\PY@tc##1{\textcolor[rgb]{0.73,0.73,0.73}{##1}}}
\@namedef{PY@tok@c}{\let\PY@it=\textit\def\PY@tc##1{\textcolor[rgb]{0.24,0.48,0.48}{##1}}}
\@namedef{PY@tok@cp}{\def\PY@tc##1{\textcolor[rgb]{0.61,0.40,0.00}{##1}}}
\@namedef{PY@tok@k}{\let\PY@bf=\textbf\def\PY@tc##1{\textcolor[rgb]{0.00,0.50,0.00}{##1}}}
\@namedef{PY@tok@kp}{\def\PY@tc##1{\textcolor[rgb]{0.00,0.50,0.00}{##1}}}
\@namedef{PY@tok@kt}{\def\PY@tc##1{\textcolor[rgb]{0.69,0.00,0.25}{##1}}}
\@namedef{PY@tok@o}{\def\PY@tc##1{\textcolor[rgb]{0.40,0.40,0.40}{##1}}}
\@namedef{PY@tok@ow}{\let\PY@bf=\textbf\def\PY@tc##1{\textcolor[rgb]{0.67,0.13,1.00}{##1}}}
\@namedef{PY@tok@nb}{\def\PY@tc##1{\textcolor[rgb]{0.00,0.50,0.00}{##1}}}
\@namedef{PY@tok@nf}{\def\PY@tc##1{\textcolor[rgb]{0.00,0.00,1.00}{##1}}}
\@namedef{PY@tok@nc}{\let\PY@bf=\textbf\def\PY@tc##1{\textcolor[rgb]{0.00,0.00,1.00}{##1}}}
\@namedef{PY@tok@nn}{\let\PY@bf=\textbf\def\PY@tc##1{\textcolor[rgb]{0.00,0.00,1.00}{##1}}}
\@namedef{PY@tok@ne}{\let\PY@bf=\textbf\def\PY@tc##1{\textcolor[rgb]{0.80,0.25,0.22}{##1}}}
\@namedef{PY@tok@nv}{\def\PY@tc##1{\textcolor[rgb]{0.10,0.09,0.49}{##1}}}
\@namedef{PY@tok@no}{\def\PY@tc##1{\textcolor[rgb]{0.53,0.00,0.00}{##1}}}
\@namedef{PY@tok@nl}{\def\PY@tc##1{\textcolor[rgb]{0.46,0.46,0.00}{##1}}}
\@namedef{PY@tok@ni}{\let\PY@bf=\textbf\def\PY@tc##1{\textcolor[rgb]{0.44,0.44,0.44}{##1}}}
\@namedef{PY@tok@na}{\def\PY@tc##1{\textcolor[rgb]{0.41,0.47,0.13}{##1}}}
\@namedef{PY@tok@nt}{\let\PY@bf=\textbf\def\PY@tc##1{\textcolor[rgb]{0.00,0.50,0.00}{##1}}}
\@namedef{PY@tok@nd}{\def\PY@tc##1{\textcolor[rgb]{0.67,0.13,1.00}{##1}}}
\@namedef{PY@tok@s}{\def\PY@tc##1{\textcolor[rgb]{0.73,0.13,0.13}{##1}}}
\@namedef{PY@tok@sd}{\let\PY@it=\textit\def\PY@tc##1{\textcolor[rgb]{0.73,0.13,0.13}{##1}}}
\@namedef{PY@tok@si}{\let\PY@bf=\textbf\def\PY@tc##1{\textcolor[rgb]{0.64,0.35,0.47}{##1}}}
\@namedef{PY@tok@se}{\let\PY@bf=\textbf\def\PY@tc##1{\textcolor[rgb]{0.67,0.36,0.12}{##1}}}
\@namedef{PY@tok@sr}{\def\PY@tc##1{\textcolor[rgb]{0.64,0.35,0.47}{##1}}}
\@namedef{PY@tok@ss}{\def\PY@tc##1{\textcolor[rgb]{0.10,0.09,0.49}{##1}}}
\@namedef{PY@tok@sx}{\def\PY@tc##1{\textcolor[rgb]{0.00,0.50,0.00}{##1}}}
\@namedef{PY@tok@m}{\def\PY@tc##1{\textcolor[rgb]{0.40,0.40,0.40}{##1}}}
\@namedef{PY@tok@gh}{\let\PY@bf=\textbf\def\PY@tc##1{\textcolor[rgb]{0.00,0.00,0.50}{##1}}}
\@namedef{PY@tok@gu}{\let\PY@bf=\textbf\def\PY@tc##1{\textcolor[rgb]{0.50,0.00,0.50}{##1}}}
\@namedef{PY@tok@gd}{\def\PY@tc##1{\textcolor[rgb]{0.63,0.00,0.00}{##1}}}
\@namedef{PY@tok@gi}{\def\PY@tc##1{\textcolor[rgb]{0.00,0.52,0.00}{##1}}}
\@namedef{PY@tok@gr}{\def\PY@tc##1{\textcolor[rgb]{0.89,0.00,0.00}{##1}}}
\@namedef{PY@tok@ge}{\let\PY@it=\textit}
\@namedef{PY@tok@gs}{\let\PY@bf=\textbf}
\@namedef{PY@tok@gp}{\let\PY@bf=\textbf\def\PY@tc##1{\textcolor[rgb]{0.00,0.00,0.50}{##1}}}
\@namedef{PY@tok@go}{\def\PY@tc##1{\textcolor[rgb]{0.44,0.44,0.44}{##1}}}
\@namedef{PY@tok@gt}{\def\PY@tc##1{\textcolor[rgb]{0.00,0.27,0.87}{##1}}}
\@namedef{PY@tok@err}{\def\PY@bc##1{{\setlength{\fboxsep}{\string -\fboxrule}\fcolorbox[rgb]{1.00,0.00,0.00}{1,1,1}{\strut ##1}}}}
\@namedef{PY@tok@kc}{\let\PY@bf=\textbf\def\PY@tc##1{\textcolor[rgb]{0.00,0.50,0.00}{##1}}}
\@namedef{PY@tok@kd}{\let\PY@bf=\textbf\def\PY@tc##1{\textcolor[rgb]{0.00,0.50,0.00}{##1}}}
\@namedef{PY@tok@kn}{\let\PY@bf=\textbf\def\PY@tc##1{\textcolor[rgb]{0.00,0.50,0.00}{##1}}}
\@namedef{PY@tok@kr}{\let\PY@bf=\textbf\def\PY@tc##1{\textcolor[rgb]{0.00,0.50,0.00}{##1}}}
\@namedef{PY@tok@bp}{\def\PY@tc##1{\textcolor[rgb]{0.00,0.50,0.00}{##1}}}
\@namedef{PY@tok@fm}{\def\PY@tc##1{\textcolor[rgb]{0.00,0.00,1.00}{##1}}}
\@namedef{PY@tok@vc}{\def\PY@tc##1{\textcolor[rgb]{0.10,0.09,0.49}{##1}}}
\@namedef{PY@tok@vg}{\def\PY@tc##1{\textcolor[rgb]{0.10,0.09,0.49}{##1}}}
\@namedef{PY@tok@vi}{\def\PY@tc##1{\textcolor[rgb]{0.10,0.09,0.49}{##1}}}
\@namedef{PY@tok@vm}{\def\PY@tc##1{\textcolor[rgb]{0.10,0.09,0.49}{##1}}}
\@namedef{PY@tok@sa}{\def\PY@tc##1{\textcolor[rgb]{0.73,0.13,0.13}{##1}}}
\@namedef{PY@tok@sb}{\def\PY@tc##1{\textcolor[rgb]{0.73,0.13,0.13}{##1}}}
\@namedef{PY@tok@sc}{\def\PY@tc##1{\textcolor[rgb]{0.73,0.13,0.13}{##1}}}
\@namedef{PY@tok@dl}{\def\PY@tc##1{\textcolor[rgb]{0.73,0.13,0.13}{##1}}}
\@namedef{PY@tok@s2}{\def\PY@tc##1{\textcolor[rgb]{0.73,0.13,0.13}{##1}}}
\@namedef{PY@tok@sh}{\def\PY@tc##1{\textcolor[rgb]{0.73,0.13,0.13}{##1}}}
\@namedef{PY@tok@s1}{\def\PY@tc##1{\textcolor[rgb]{0.73,0.13,0.13}{##1}}}
\@namedef{PY@tok@mb}{\def\PY@tc##1{\textcolor[rgb]{0.40,0.40,0.40}{##1}}}
\@namedef{PY@tok@mf}{\def\PY@tc##1{\textcolor[rgb]{0.40,0.40,0.40}{##1}}}
\@namedef{PY@tok@mh}{\def\PY@tc##1{\textcolor[rgb]{0.40,0.40,0.40}{##1}}}
\@namedef{PY@tok@mi}{\def\PY@tc##1{\textcolor[rgb]{0.40,0.40,0.40}{##1}}}
\@namedef{PY@tok@il}{\def\PY@tc##1{\textcolor[rgb]{0.40,0.40,0.40}{##1}}}
\@namedef{PY@tok@mo}{\def\PY@tc##1{\textcolor[rgb]{0.40,0.40,0.40}{##1}}}
\@namedef{PY@tok@ch}{\let\PY@it=\textit\def\PY@tc##1{\textcolor[rgb]{0.24,0.48,0.48}{##1}}}
\@namedef{PY@tok@cm}{\let\PY@it=\textit\def\PY@tc##1{\textcolor[rgb]{0.24,0.48,0.48}{##1}}}
\@namedef{PY@tok@cpf}{\let\PY@it=\textit\def\PY@tc##1{\textcolor[rgb]{0.24,0.48,0.48}{##1}}}
\@namedef{PY@tok@c1}{\let\PY@it=\textit\def\PY@tc##1{\textcolor[rgb]{0.24,0.48,0.48}{##1}}}
\@namedef{PY@tok@cs}{\let\PY@it=\textit\def\PY@tc##1{\textcolor[rgb]{0.24,0.48,0.48}{##1}}}

\def\PYZbs{\char`\\}
\def\PYZus{\char`\_}
\def\PYZob{\char`\{}
\def\PYZcb{\char`\}}
\def\PYZca{\char`\^}
\def\PYZam{\char`\&}
\def\PYZlt{\char`\<}
\def\PYZgt{\char`\>}
\def\PYZsh{\char`\#}
\def\PYZpc{\char`\%}
\def\PYZdl{\char`\$}
\def\PYZhy{\char`\-}
\def\PYZsq{\char`\'}
\def\PYZdq{\char`\"}
\def\PYZti{\char`\~}
% for compatibility with earlier versions
\def\PYZat{@}
\def\PYZlb{[}
\def\PYZrb{]}
\makeatother


    % For linebreaks inside Verbatim environment from package fancyvrb.
    \makeatletter
        \newbox\Wrappedcontinuationbox
        \newbox\Wrappedvisiblespacebox
        \newcommand*\Wrappedvisiblespace {\textcolor{red}{\textvisiblespace}}
        \newcommand*\Wrappedcontinuationsymbol {\textcolor{red}{\llap{\tiny$\m@th\hookrightarrow$}}}
        \newcommand*\Wrappedcontinuationindent {3ex }
        \newcommand*\Wrappedafterbreak {\kern\Wrappedcontinuationindent\copy\Wrappedcontinuationbox}
        % Take advantage of the already applied Pygments mark-up to insert
        % potential linebreaks for TeX processing.
        %        {, <, #, %, $, ' and ": go to next line.
        %        _, }, ^, &, >, - and ~: stay at end of broken line.
        % Use of \textquotesingle for straight quote.
        \newcommand*\Wrappedbreaksatspecials {%
            \def\PYGZus{\discretionary{\char`\_}{\Wrappedafterbreak}{\char`\_}}%
            \def\PYGZob{\discretionary{}{\Wrappedafterbreak\char`\{}{\char`\{}}%
            \def\PYGZcb{\discretionary{\char`\}}{\Wrappedafterbreak}{\char`\}}}%
            \def\PYGZca{\discretionary{\char`\^}{\Wrappedafterbreak}{\char`\^}}%
            \def\PYGZam{\discretionary{\char`\&}{\Wrappedafterbreak}{\char`\&}}%
            \def\PYGZlt{\discretionary{}{\Wrappedafterbreak\char`\<}{\char`\<}}%
            \def\PYGZgt{\discretionary{\char`\>}{\Wrappedafterbreak}{\char`\>}}%
            \def\PYGZsh{\discretionary{}{\Wrappedafterbreak\char`\#}{\char`\#}}%
            \def\PYGZpc{\discretionary{}{\Wrappedafterbreak\char`\%}{\char`\%}}%
            \def\PYGZdl{\discretionary{}{\Wrappedafterbreak\char`\$}{\char`\$}}%
            \def\PYGZhy{\discretionary{\char`\-}{\Wrappedafterbreak}{\char`\-}}%
            \def\PYGZsq{\discretionary{}{\Wrappedafterbreak\textquotesingle}{\textquotesingle}}%
            \def\PYGZdq{\discretionary{}{\Wrappedafterbreak\char`\"}{\char`\"}}%
            \def\PYGZti{\discretionary{\char`\~}{\Wrappedafterbreak}{\char`\~}}%
        }
        % Some characters . , ; ? ! / are not pygmentized.
        % This macro makes them "active" and they will insert potential linebreaks
        \newcommand*\Wrappedbreaksatpunct {%
            \lccode`\~`\.\lowercase{\def~}{\discretionary{\hbox{\char`\.}}{\Wrappedafterbreak}{\hbox{\char`\.}}}%
            \lccode`\~`\,\lowercase{\def~}{\discretionary{\hbox{\char`\,}}{\Wrappedafterbreak}{\hbox{\char`\,}}}%
            \lccode`\~`\;\lowercase{\def~}{\discretionary{\hbox{\char`\;}}{\Wrappedafterbreak}{\hbox{\char`\;}}}%
            \lccode`\~`\:\lowercase{\def~}{\discretionary{\hbox{\char`\:}}{\Wrappedafterbreak}{\hbox{\char`\:}}}%
            \lccode`\~`\?\lowercase{\def~}{\discretionary{\hbox{\char`\?}}{\Wrappedafterbreak}{\hbox{\char`\?}}}%
            \lccode`\~`\!\lowercase{\def~}{\discretionary{\hbox{\char`\!}}{\Wrappedafterbreak}{\hbox{\char`\!}}}%
            \lccode`\~`\/\lowercase{\def~}{\discretionary{\hbox{\char`\/}}{\Wrappedafterbreak}{\hbox{\char`\/}}}%
            \catcode`\.\active
            \catcode`\,\active
            \catcode`\;\active
            \catcode`\:\active
            \catcode`\?\active
            \catcode`\!\active
            \catcode`\/\active
            \lccode`\~`\~
        }
    \makeatother

    \let\OriginalVerbatim=\Verbatim
    \makeatletter
    \renewcommand{\Verbatim}[1][1]{%
        %\parskip\z@skip
        \sbox\Wrappedcontinuationbox {\Wrappedcontinuationsymbol}%
        \sbox\Wrappedvisiblespacebox {\FV@SetupFont\Wrappedvisiblespace}%
        \def\FancyVerbFormatLine ##1{\hsize\linewidth
            \vtop{\raggedright\hyphenpenalty\z@\exhyphenpenalty\z@
                \doublehyphendemerits\z@\finalhyphendemerits\z@
                \strut ##1\strut}%
        }%
        % If the linebreak is at a space, the latter will be displayed as visible
        % space at end of first line, and a continuation symbol starts next line.
        % Stretch/shrink are however usually zero for typewriter font.
        \def\FV@Space {%
            \nobreak\hskip\z@ plus\fontdimen3\font minus\fontdimen4\font
            \discretionary{\copy\Wrappedvisiblespacebox}{\Wrappedafterbreak}
            {\kern\fontdimen2\font}%
        }%

        % Allow breaks at special characters using \PYG... macros.
        \Wrappedbreaksatspecials
        % Breaks at punctuation characters . , ; ? ! and / need catcode=\active
        \OriginalVerbatim[#1,codes*=\Wrappedbreaksatpunct]%
    }
    \makeatother

    % Exact colors from NB
    \definecolor{incolor}{HTML}{303F9F}
    \definecolor{outcolor}{HTML}{D84315}
    \definecolor{cellborder}{HTML}{CFCFCF}
    \definecolor{cellbackground}{HTML}{F7F7F7}

    % prompt
    \makeatletter
    \newcommand{\boxspacing}{\kern\kvtcb@left@rule\kern\kvtcb@boxsep}
    \makeatother
    \newcommand{\prompt}[4]{
        {\ttfamily\llap{{\color{#2}[#3]:\hspace{3pt}#4}}\vspace{-\baselineskip}}
    }
    

    
    % Prevent overflowing lines due to hard-to-break entities
    \sloppy
    % Setup hyperref package
    \hypersetup{
      breaklinks=true,  % so long urls are correctly broken across lines
      colorlinks=true,
      urlcolor=urlcolor,
      linkcolor=linkcolor,
      citecolor=citecolor,
      }
    % Slightly bigger margins than the latex defaults
    
    \geometry{verbose,tmargin=1in,bmargin=1in,lmargin=1in,rmargin=1in}
    
    

\begin{document}
    
    \maketitle
    
    

    
    \section{The exam}\label{the-exam}

This document describes the idea behind the test for the module 1 of the
Software and Computing course, and the module 2 for the Applied Physics
branch of the same course.

    \subsection{general guidelines}\label{general-guidelines}

\begin{itemize}
\tightlist
\item
  You will be evaluated on a programming project;
\item
  this project needs to be hosted on a public repository;
\item
  you are free to choose the programming language of the project;
\item
  accepted control version systems are git and fossil;
\end{itemize}

    \subsection{topics of evaluation}\label{topics-of-evaluation}

The project must be executable on a different machine from the one you
have developed it into. Limitations can be put on the operative system
and platform version required, but if so they must be explicitely
specified.

\begin{itemize}
\tightlist
\item
  clarity of the repository commit history
\item
  organization and clarity of the source code
\item
  organization and clarity of the documentation
\item
  presence and executability of test routines
\end{itemize}

    \subsubsection{optional bonuses (don't rely on
them)}\label{optional-bonuses-dont-rely-on-them}

\begin{itemize}
\tightlist
\item
  Usage of innovative technologies and libraries
\item
  Contribution to open source projects
\end{itemize}

    \subsubsection{exam progression}\label{exam-progression}

Please start discussing your exam project \textbf{before} starting to
work on it. Discussing it early will avoid wasting effort and time for
something that is not appropriate for the exam!

possible projects: * an idea that you find interesting (please discuss
it first) * deciding a project together * Sharing the project with other
exams or thesis

    \subsection{The spirit of the exam}\label{the-spirit-of-the-exam}

The goal of the exam IS NOT to assess if you can write a very
complicated simulation.

The goal is to show that you are able to write a software that can be
used by other researcher and that it can pass the ``bus test'': if
tomorrow you are unable to work on this project, would other people be
able to keep it alive?

    To fullfill this goal, it is not useful to perform virtuosisms of
computational skills. The project does not have to be complicated. It
has to be reliable and understandable by others.

This other person should not be burdained of effort to understand your
code.

\textbf{``Go and read the source code''} it's not a reasonable answer.

\textbf{``Go and edit the source code''} to use it is even less so.

    They should be able to install your code, run it and replicate your
results.

consider giving the software a command line interface that can be run
with some parameters, rather than having to explicitely run a bunch of
scripts in a non explicit sequence.

    \subsection{General guidelines on the various topics of
evaluation}\label{general-guidelines-on-the-various-topics-of-evaluation}

\subsubsection{Documentation}\label{documentation}

Documentation should be written in English, with proper sentences and
coherent structure. The whole documentation should be accessible from
the landing page of the repository, either in the readme or using
hypertextual links: the user should not be left guessing on where to
find it.

    The documentation should explain: * what problem is the function or
program trying to solve * what are the parameters that are exposed to
the user and, if there are defaults parameters, what they are * how it
is implemented (does it use specific algorithms? do have any peculiar
assumption about the input data?)

    If possible the documentation should also include examples of execution,
configurations and results, so that the user can understand the basic
uses of the program/library, and could test on their own.

a good reference for writing documentation is
https://www.divio.com/blog/documentation/

    \subsubsection{Testing}\label{testing}

Testing should involve all the functions exposed to the user, and should
ideally cover any combination of parameters. This means that in general
is a good idea to keep the functions simple and composable, to avoid
overly brittle and complicated test suites.

The test name should be indicative of what the test is trying to
achieve. \texttt{test\_1} is not a clear name
\texttt{test\_function\_is\_even} is more clear.

Is possible, write documentations (docstrings) for your tests as well,
following the principle of explaining WHY the test is being performed,
alongside WHAT is testing.

    \textbf{REMEMBER THAT FLOATING POINT NUMBERS HAVE ISSUES WITH EQUALITY
TESTING!!!}

Make sure that any test could only fail for a single reason: if a test
have several failing conditions, it will be difficult to pinpoint the
exact cause.

    Every time you find a bug, write an associated test \textbf{before}
solving the bug, make sure that it fails and then solve it (making it
sure that now it passes). This will prevent the bug from showing again
in the future.

Tests are a way to also establish the interface of your program. They
are a promise to your users: if something is tested it should work. This
means that every commit should be tested and only submitted if
\textbf{all} the tests passes.

    Even if not compulsory, I strongly suggest to use a coverage test system
(with libraries such as
\href{https://coverage.readthedocs.io/en/coverage-5.0.3/}{\emph{coverage.py}})
and try to achieve 100\% coverage.

test code should be well separated from the library (or simulation)
code, and definitely not mixed in the same file.

testing does not need to verify that the code can be run (this should be
taken for granted), but that upon executino the results are the one
expected. This is the reason to use asserts.

    \paragraph{visualization functions - no testing
needed}\label{visualization-functions---no-testing-needed}

Your code might contains faunctions that, given some data in input, will
generate some visualization of those data.

In general, it's accepted that this kind of functions don't require
testing, as the effort necessary to have a robust test will overwhelm
any benefit.

Of course, a function of this kind should not contain ANY PREPROCESSING
of the data, but just the visualization itself.

    \paragraph{Testing functions and
printing}\label{testing-functions-and-printing}

Testing function should not print or return anything, as this would only
confuse the result of the testing suite.

\paragraph{Use of Hypothesis}\label{use-of-hypothesis}

The use of the hypothesis library is welcome, but only once there is a
proper test suite in place with more simple tools such as unittest or
pytest

    \subsubsection{Code organization and
clarity}\label{code-organization-and-clarity}

Follow a consistent programming style. A good starting point is the
\href{https://www.python.org/dev/peps/pep-0008/}{PEP8}. There are many
guidelines out there, pick one and follow it.

Use proper naming, both explanatory and consistent. This helps the user
using your library. If in one part of the library you have functions
such as \texttt{has\_property}, don't change the style in another part
to something like \texttt{is\_something}

    Avoid dependency from global states as much as possible. The only
tolerated exception is for configuration values that are never modified.
Do not write on the global state. No, yours is not a special enough
case.

    leverage the power of functions to avoid useless code repetition and
make your code easier to read for someone new to it. Functions should
ideally do a single thing: huge number of parameters in the call is
usually indicative that the function is trying to do to much.

functions should be decorrelated, meaning that there should not be a
fixed order to call them otherwise weird error arises. If a group of
functions needs to be called in a fixed order, you are probably trying
to write an object without realizing it. In some cases this can be
simplified using
\href{https://docs.python.org/3/library/functools.html\#functools.partial}{\emph{partial}}
functions and
\href{https://docs.python.org/3/library/contextlib.html\#contextlib.contextmanager}{\emph{contextmanager}}
decorators.

    If your code dependes on external libraries, remember to point out which
ones, and which is the minimum (and possibly maximum) version that are
necessary. In general try to balance out how many libraries your code
dependes on. Some libraries are commonly used by most users in the same
community, and those are not a problem (such as numpy, scipy, pandas and
scikit.learn for the scientific python community), but try to limit
dependencies from \emph{unorthodox} libraries

    if you need to connect to online services, consider
\href{https://docs.python.org/3/library/functools.html\#functools.lru_cache}{\emph{caching}}
or, at least, minimize the number of connections used to avoid bogging
them down.

\textbf{Make sure that there are no reference to specifi paths of your
computer in the code!} Asking the library to load a file in the
directory
\texttt{\textbackslash{}home\textbackslash{}myname\textbackslash{}myprojects\textbackslash{}SandCexam}
is unacceptable. It is suggested to use a third party continuous
integration service such as \href{https://travis-ci.org/}{TravisCI}, as
it will show most of these brittle errors. I need to be able to run the
software on my computer, this is a prerequisite, and as such this kind
of mistake precludes the project to receive and evaluation!

    \subsubsection{comments}\label{comments}

Remember to comment the code: comments should indicate WHY the code is
doing something, not describe WHAT is doing. Proper function and
variable naming can help reduce the needs for comments.

bad:

\begin{Shaded}
\begin{Highlighting}[]
\NormalTok{lenght }\OperatorTok{=} \BuiltInTok{min}\NormalTok{(lenght, }\DecValTok{80}\NormalTok{)}
\end{Highlighting}
\end{Shaded}

    better:

\begin{Shaded}
\begin{Highlighting}[]
\CommentTok{\# limit the text lenght to avoid overflow issues}
\NormalTok{lenght }\OperatorTok{=} \BuiltInTok{min}\NormalTok{(lenght, }\DecValTok{80}\NormalTok{)}
\end{Highlighting}
\end{Shaded}

    best:

\begin{Shaded}
\begin{Highlighting}[]
\CommentTok{\# limit the text lenght to avoid overflow issues}
\NormalTok{terminal\_character\_limit }\OperatorTok{=} \DecValTok{80}
\NormalTok{text\_line\_lenght }\OperatorTok{=} \BuiltInTok{min}\NormalTok{(text\_line\_lenght, }
\NormalTok{                       terminal\_character\_limit)}
\end{Highlighting}
\end{Shaded}

    \subsubsection{assert and exceptions}\label{assert-and-exceptions}

In python don't be overly defensive with testing the input for a
function. If your function need to insert some values in a dict, you
don't need to test that the input is a dict, but only that is a mappable
(does have a \texttt{set\_item} function).

Even better, don't try to verify beforehand but rather use a try except
clause in the code and explain the problem to the user.

    \textbf{Don't use \texttt{assert} to check for important conditions},
but use if-else clause and raise an approriate exception (subclassing
one of the existing one, if needed) if the condition is not valid.

Asserts are supposed to be used at the end of the code to make the code
invariant explicit (for example that the two returned arrays are of the
same lenght). In a reasonable condition one should never expect to see
one of these \texttt{assert} fail: they are used to communicate
infomations to the reader.

    bad:

\begin{Shaded}
\begin{Highlighting}[]
\KeywordTok{def}\NormalTok{ add\_key(mappable, key, value):}
    \ControlFlowTok{assert} \BuiltInTok{isinstance}\NormalTok{(mappable, }\BuiltInTok{dict}\NormalTok{)}
\NormalTok{    mappable[key] }\OperatorTok{=}\NormalTok{ value}
\end{Highlighting}
\end{Shaded}

    better:

\begin{Shaded}
\begin{Highlighting}[]
\KeywordTok{def}\NormalTok{ add\_key(mappable, key, value):}
    \ControlFlowTok{if} \KeywordTok{not} \BuiltInTok{isinstance}\NormalTok{(mappable, }\BuiltInTok{dict}\NormalTok{):}
        \ControlFlowTok{raise} \PreprocessorTok{TypeError}\NormalTok{(}\StringTok{"the mappable object should be a dict"}\NormalTok{)}
\NormalTok{    mappable[key] }\OperatorTok{=}\NormalTok{ value}
\end{Highlighting}
\end{Shaded}

    even better:

\begin{Shaded}
\begin{Highlighting}[]
\KeywordTok{def}\NormalTok{ add\_key(mappable, key, value):}
    \ControlFlowTok{try}\NormalTok{:}
\NormalTok{        mappable[key] }\OperatorTok{=}\NormalTok{ value}
    \ControlFlowTok{except} \PreprocessorTok{TypeError} \ImportTok{as}\NormalTok{ e:}
\NormalTok{        err }\OperatorTok{=} \StringTok{"object does not support item assignment"}
        \ControlFlowTok{if}\NormalTok{ err }\KeywordTok{in}\NormalTok{ e.args[}\DecValTok{0}\NormalTok{]:}
\NormalTok{            msg }\OperatorTok{=} \StringTok{"}\SpecialCharTok{\{\}}\StringTok{ doesn\textquotesingle{}t support item assignment"}
\NormalTok{            msg }\OperatorTok{=}\NormalTok{ msg.}\BuiltInTok{format}\NormalTok{(}\BuiltInTok{type}\NormalTok{(mappable))}
            \ControlFlowTok{raise} \PreprocessorTok{TypeError}\NormalTok{(msg) }\ImportTok{from} \VariableTok{None}
\end{Highlighting}
\end{Shaded}

    best:

\begin{Shaded}
\begin{Highlighting}[]
\KeywordTok{def}\NormalTok{ add\_key(mappable, key, value):}
\NormalTok{    mappable[key] }\OperatorTok{=}\NormalTok{ value}
\end{Highlighting}
\end{Shaded}

    \subsubsection{Version control commits}\label{version-control-commits}

\textbf{DO NOT UPLOAD PASSOWORDS OR ANY OTHER SECURITY INFORMATION!!!}
This includes also the ip address of servers, usernames, personal
informations or anything of the sort. Include them in a separate text
file if you really need them and only include in the commit a dummy file
with fake informations.

All commits should have a brief and descriptive title. ``some edits'' is
not a good description.

    Each commit should be a single kind of edit. Editing a function might
need to update also the documentation and testing, and possibly other
code that relies on that function. But don't mix random extensions of
the documentation, code editing, and so on.

try to avoid including temporary files (such as the
\texttt{\_\_pycache\_\_} folder) in the commits, as they make the
repository bigger and harder to understand.

    avoid including very big files in the repository, unless necessary.
Consider the possibility of dynamically generate them during the
installation or the first run.

Having multiple smaller commits is absolutely ok. Projects with hundreds
of commits are absolutely normal, don't be afraid to commit multiple
time a day.

    Do not upload datasets, especially if they are from an experiment that
you're collaborating with, as you might have copyright and intellectual
property issues.

    \subsection{Kind of possible projects and gotchas of each
one}\label{kind-of-possible-projects-and-gotchas-of-each-one}

\begin{itemize}
\tightlist
\item
  libraries
\item
  simulations
\end{itemize}

these are not the only possible ones (frameworks, web services,
etc\ldots{} would be accepted upon discussion), but they are the vast
majority of the projects.

    \subsection{management of simulations}\label{management-of-simulations}

If the project is a simulation, the main goal is to be able to replicate
the results, and to batch execute them.

\begin{itemize}
\tightlist
\item
  there should be no hard dependency on pathways or files from your
  computer
\item
  random elements should be managed explicitely using a random seed
\end{itemize}

    \begin{itemize}
\tightlist
\item
  the simulation, analysis and plotting steps should be separated, so
  that the analysis can be repeated without performing the simulation
  again. Ideally should be possible performing batch analysis of the
  results of several simulations.
\item
  different configurations for the simulation should be included in a
  separate configuration file, that could be provided by the user
\end{itemize}

    \subsection{management of libraries}\label{management-of-libraries}

if the project is a library, the main goal is for a user to be able to
use it in their own libraries and simulations.

\begin{itemize}
\tightlist
\item
  the API of the library (i.e.~the functions and how they are related)
  should be properly designed and explained
\item
  include examples and tutorial in the documentation
\end{itemize}

    \begin{itemize}
\tightlist
\item
  configuration of the library should be transparent to the user, and
  should not rely on weird system states
\item
  be clear on the algorithms used and possible limitations of the method
  (memory, disk space, parallelism, etc\ldots)
\end{itemize}

    \subsection{Jupyter notebooks}\label{jupyter-notebooks}

Do not use Jupyter notebooks to store functions, classes and testing.

Palce them in separate files and import them in the notebook.

Jupyter notebooks are not designed for lont term, clean code storage,
but rather to keep trace of the execution of said code. They are ideal
for tutorial, how-to and similar, and their use is welcome to show an
actual execution of your code.

    \subsection{examples of repositories from other
students}\label{examples-of-repositories-from-other-students}

these are some repositories that obtained the full mark in the exam (for
the applied physics course), and could be used as a reference.

\begin{itemize}
\tightlist
\item
  \url{https://github.com/JonathanFrassineti/Software-Project}
\item
  \url{https://github.com/filippocastelli/KEGGutils}
\item
  \url{https://github.com/riccardoscheda/AnomalousDiffusion}
\end{itemize}


    % Add a bibliography block to the postdoc
    
    
    
\end{document}
